\documentclass[12pt, titlepage]{article}
\usepackage{beamerarticle}
\usepackage[utf8]{inputenc}
\usepackage{amssymb,amsmath}

% Page settings
\usepackage[letterpaper, margin=1in]{geometry}
\usepackage{times}
%\usepackage{palatino}%\usepackage{lmodern, times}
%\usepackage{lmodern}%\usepackage{lmodern, times}
\usepackage{setspace}                           
\onehalfspacing %\doublespacing  % \singlespacing 

% Appendix
\usepackage{appendix}

% Line numbers
%\usepackage{lineno}
%\linenumbers

% Links
\usepackage{hyperref}
\hypersetup{%
  colorlinks=false,% hyperlinks will be black
  linkbordercolor=red,% hyperlink borders will be red
  pdfborderstyle={/S/U/W 1}% border style will be underline of width 1pt
}

% Tables
\usepackage{array,booktabs,longtable,rotating}
\newenvironment{tablenotes}[1][]{
  \begin{minipage}{\textwidth}\emph{Notes:}{\footnotesize #1}
}{\end{minipage}}
\makeatletter
\def\fps@table{htbp}
\makeatother

% Graphics
\usepackage{graphicx,grffile}
\makeatletter
\def\maxwidth{\ifdim\Gin@nat@width>\linewidth\linewidth\else\Gin@nat@width\fi}
\def\maxheight{\ifdim\Gin@nat@height>\textheight\textheight\else\Gin@nat@height\fi}
\makeatother
% Scale images if necessary, so that they will not overflow the page
% margins by default, and it is still possible to overwrite the defaults
% using explicit options in \includegraphics[width, height, ...]{}
\setkeys{Gin}{width=\maxwidth,height=\maxheight,keepaspectratio}
% set default figure placement to htbp
\makeatletter
\def\fps@figure{htbp}
\makeatother

\usepackage{natbib}% plainnat
\bibliographystyle{abbrvnat}
\setcitestyle{authoryear,open={(},close={)}}
%\bibliographystyle{aer}


\setlength{\emergencystretch}{3em}  % prevent overfull lines
\providecommand{\tightlist}{%
  \setlength{\itemsep}{0pt}\setlength{\parskip}{0pt}}



\institute{}
\titlegraphic{}

\usepackage{xcolor}
\newcommand\todonote[1]{\textcolor{red}{#1}}



% Subscript
\newcommand\sub[1]{_{#1}}
\newcommand\supsc[1]{^{#1}}

\title{Gender Imbalance Online\thanks{Blasco: Harvard Institute for Quantitative Social Science, Harvard
University, 1737 Cambridge Street, Cambridge, MA 02138 (email:
\href{mailto:ablasco@fas.harvard.edu}{\nolinkurl{ablasco@fas.harvard.edu}}).
Lakhani: Harvard Business School, Soldiers Field, Boston, MA 02163, and
National Bureau of Economic Research (email:
\href{mailto:k@hbs.edu}{\nolinkurl{k@hbs.edu}}). Menietti: Harvard
Institute for Quantitative Social Science, Harvard University, 1737
Cambridge Street, Cambridge, MA 02138 (email:
\href{mailto:mmenietti@fas.harvard.edu}{\nolinkurl{mmenietti@fas.harvard.edu}}).
We gratefully acknowledge the financial support of XXXX.}}
\author{Andrea Blasco \and Karim R. Lakhani \and Michael Menietti}
\date{Last updated: 09 May, 2017}

\begin{document}
\maketitle
\begin{abstract}
We report results of a natural field experiment conducted XXXX.

\smallskip\noindent 
JEL Classification: XXXX.

\smallskip\noindent 
Keywords: XXXX.
\end{abstract}


% Todo notes
%\usepackage[textsize=tiny]{todonotes}
%\newcommand\redmarginpar[1]{\marginpar{\footnotesize{\textcolor{red}{#1}}}}
%\listoftodos[Notes]
\clearpage

\section{Introduction}\label{introduction}

Though the proportion of Internet users is nearly the same across the
genders, researchers have found significant gender imbalances in online
behavior:

\begin{itemize}
\tightlist
\item
  Only 13 percent of Wikipedia contributors are women
  \citep{hill2013wikipedia};
\item
  Wikipedia articles on women are more likely to be missing than are
  articles on men relative to Britannica \citep{reagle2011gender};
\item
  Less than 5 percent of StackOverflow users answering technical
  questions are women \citep{vasilescu2012gender};
\item
  Women make a small fraction of people going to hackathons or taking
  part in online competitions on crowdsourcing platforms (Innocentive,
  TopCoder, Kaggle).
\end{itemize}

The reasons behind this gender gap are not fully understood. It is clear
that it is not due to discriminatory rules put in place on these
platforms. Rather the culprit seems to be a mix of psychological
motivations and institutional characteristics of the platforms that
might have inadvertently promoted an imbalance.

We conjecture that an important mechanism of imbalance could be the
combination of

\begin{enumerate}
\def\labelenumi{\arabic{enumi}.}
\tightlist
\item
  Gamification \& incentives
\item
  Differences in preferences between the genders
\end{enumerate}

Online platforms use elements of game play to engage their members
(rankings, game points, competition with others, etc.) and stimulate
higher levels of user activity. If men and women have different
inclinations towards these incentives, they might inadvertently
stimulate an imbalanced participation.

\section{Literature}\label{literature}

\subsection{Gender composition}\label{gender-composition}

\citet{charness2011gender} finds that males cooperate less often when
observed by groups of peers of the same gender, while females cooperate
more often. (Possible interpretation: males and females want to signal
different ``skills'' to their peers).

\citet{de2015gender} shows evidence of gender discrimination in
committees for haring academics in Italy. Mixed-gender committees are
less likely xxx.

\subsection{Role models}\label{role-models}

According to the Oxford dictionary, a ``role model'' is a person ``whose
behaviour, attitudes, and/or values are seen by individuals as worthy of
imitation or emulation.'' In economics, role models are a special case
of ``externality.'' It is also related with the notion of ``peer
effects,'' as effects xxx. Although by looking at what defines a peer, a
role model may not fall in this definition.

Theory. See Durlauf, Bisin, Verdie, Young, etc. Transmission of values.

\subsubsection{Education}\label{education}

\citet{carrell2010sex} Air Force Academy study

\citet{lyle2007estimating} examines a natural experiment at West Point
and finds a positive effect of senior cadets' specialization on junior's
future decisions about specialization.

For \citet{gilbert1985dimensions}, female students give relatively more
importance to their role-model relationships to their professional
development than male students. {[}Why so?{]}

\citet{dee2004teachers} examines the role of teachers {[}results TBA{]}

\subsubsection{Computer science}\label{computer-science}

\citet{cheryan2013enduring} finds that seemingly small interventions,
such as changing decorations in CS room at school/university, can have
an impact on women's careers in CS.

\subsubsection{Women}\label{women}

\citet{marx2002female} finds that female experimenters administering a
math test can have a positive impact on women's performance in the test.
Likewise, \citet{latu2013successful} finds that even a subtle exposure
to pictures of highly successful female role models (such as Hillary
Clinton or Angela Merkel) can have a positive impact on women's
behavior.

\citet{glynn2015identifying} finds that having daughters affects judges
decisions.

\citet{marx2012superstars} argues that role model effectiveness can be
enhanced via increasing similarity among in-group role models. {[}Notes:
in psychology, similarity is i) shared identity e.g., gender, race ii)
shared attributes, e.g., backgrounds, interests{]}

\citep[ p.~809]{marx2012superstars} have three questions with 1 to 7
scales 1. ``How similar do you perceive the candidate to be to you?'' 2.
``How competent do you think the candidate is in math?'' 3. ``The
candidate is inspirational for me.''

\subsubsection{Evaluations}\label{evaluations}

\citet{bohnet2015performance}

\section{Experimental design}\label{experimental-design}

\subsection{Profiles selection}\label{profiles-selection}

\begin{enumerate}
\def\labelenumi{\arabic{enumi}.}
\tightlist
\item
  Recruitment profiles\ldots{}
\item
  Consent profiles\ldots{}
\item
  Profile survey\ldots{}
\end{enumerate}

Evaluation at CLER

\begin{enumerate}
\def\labelenumi{\arabic{enumi}.}
\tightlist
\item
  Recruitment\ldots{}
\item
  Consent for CLER\ldots{}
\end{enumerate}

\subsection{Preliminary/final survey}\label{preliminaryfinal-survey}

\begin{enumerate}
\def\labelenumi{\arabic{enumi}.}
\tightlist
\item
  Recruit survey\ldots{}
\item
  Survey questions\ldots{}
\end{enumerate}

\subsection{Treatment}\label{treatment}

\begin{enumerate}
\def\labelenumi{\arabic{enumi}.}
\tightlist
\item
  solicitation\ldots{}
\item
  debriefing\ldots{}
\end{enumerate}

AER Registration

\subsection{Power simulations}\label{power-simulations}

\subsection{Context}\label{context}

We designed two interventions in collaboration with HeroX.com, a
crowdsourcing platform. We view HeroX as an example of a competitive
(platform users make submissions solving a given problem and the top
submissions are awarded a cash prize) and collaborative environment,
respectively.

\subsection{Data}\label{data}

\subsection{Creating role-model
profiles}\label{creating-role-model-profiles}

The challenge is to make realisitic role models and xxx. We recruit 19
members who have won a previous challenge: 9 men and 10 women.

We begin with bio descriptions provided by each of these members. We
restrict to people with a high quality profile description that xxxx
represantitive and a profile picture that can be used to xxxx (on this
basis we exclude 9 profiles).

We create high quality and low quality pool of bios. The choice of names
is crucial to the experiment. Bertrand and Mullainathan: - tabulate
historical names by race - they used survey to validate their choices

\begin{itemize}
\tightlist
\item
  High-low quality and Names
\end{itemize}

\renewcommand\refname{References}
\bibliography{library.bib}

\end{document}