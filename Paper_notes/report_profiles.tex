\documentclass[12pt, titlepage]{article}
\usepackage{beamerarticle}
\usepackage[utf8]{inputenc}
\usepackage{hyperref}
\usepackage{amssymb,amsmath}

% Page settings
\usepackage[letterpaper, margin=1in]{geometry}
\usepackage{times}
%\usepackage{palatino}%\usepackage{lmodern, times}
%\usepackage{lmodern}%\usepackage{lmodern, times}
\usepackage{setspace}                           
\onehalfspacing %\doublespacing  % \singlespacing 

% Appendix
\usepackage{appendix}

% Line numbers
%\usepackage{lineno}
%\linenumbers


% Tables
\usepackage{array,booktabs,longtable,rotating}
\newenvironment{tablenotes}[1][]{
  \begin{minipage}{\textwidth}\emph{Notes:}{\footnotesize #1}
}{\end{minipage}}
\makeatletter
\def\fps@table{htbp}
\makeatother

% Graphics
\usepackage{graphicx,grffile}
\makeatletter
\def\maxwidth{\ifdim\Gin@nat@width>\linewidth\linewidth\else\Gin@nat@width\fi}
\def\maxheight{\ifdim\Gin@nat@height>\textheight\textheight\else\Gin@nat@height\fi}
\makeatother
% Scale images if necessary, so that they will not overflow the page
% margins by default, and it is still possible to overwrite the defaults
% using explicit options in \includegraphics[width, height, ...]{}
\setkeys{Gin}{width=\maxwidth,height=\maxheight,keepaspectratio}
% set default figure placement to htbp
\makeatletter
\def\fps@figure{htbp}
\makeatother

\usepackage{natbib}% plainnat
\bibliographystyle{abbrvnat}
\setcitestyle{authoryear,open={(},close={)}}
%\bibliographystyle{aer}
\usepackage{color}
\usepackage{fancyvrb}
\newcommand{\VerbBar}{|}
\newcommand{\VERB}{\Verb[commandchars=\\\{\}]}
\DefineVerbatimEnvironment{Highlighting}{Verbatim}{commandchars=\\\{\}}
% Add ',fontsize=\small' for more characters per line
\usepackage{framed}
\definecolor{shadecolor}{RGB}{255,255,255}
\newenvironment{Shaded}{\begin{snugshade}}{\end{snugshade}}
\newcommand{\KeywordTok}[1]{\textcolor[rgb]{0.12,0.11,0.11}{\textbf{#1}}}
\newcommand{\DataTypeTok}[1]{\textcolor[rgb]{0.00,0.34,0.68}{#1}}
\newcommand{\DecValTok}[1]{\textcolor[rgb]{0.69,0.50,0.00}{#1}}
\newcommand{\BaseNTok}[1]{\textcolor[rgb]{0.69,0.50,0.00}{#1}}
\newcommand{\FloatTok}[1]{\textcolor[rgb]{0.69,0.50,0.00}{#1}}
\newcommand{\ConstantTok}[1]{\textcolor[rgb]{0.67,0.33,0.00}{#1}}
\newcommand{\CharTok}[1]{\textcolor[rgb]{0.57,0.30,0.62}{#1}}
\newcommand{\SpecialCharTok}[1]{\textcolor[rgb]{0.24,0.68,0.91}{#1}}
\newcommand{\StringTok}[1]{\textcolor[rgb]{0.75,0.01,0.01}{#1}}
\newcommand{\VerbatimStringTok}[1]{\textcolor[rgb]{0.75,0.01,0.01}{#1}}
\newcommand{\SpecialStringTok}[1]{\textcolor[rgb]{1.00,0.33,0.00}{#1}}
\newcommand{\ImportTok}[1]{\textcolor[rgb]{1.00,0.33,0.00}{#1}}
\newcommand{\CommentTok}[1]{\textcolor[rgb]{0.54,0.53,0.53}{#1}}
\newcommand{\DocumentationTok}[1]{\textcolor[rgb]{0.38,0.47,0.50}{#1}}
\newcommand{\AnnotationTok}[1]{\textcolor[rgb]{0.79,0.38,0.79}{#1}}
\newcommand{\CommentVarTok}[1]{\textcolor[rgb]{0.00,0.58,1.00}{#1}}
\newcommand{\OtherTok}[1]{\textcolor[rgb]{0.00,0.43,0.16}{#1}}
\newcommand{\FunctionTok}[1]{\textcolor[rgb]{0.39,0.29,0.61}{#1}}
\newcommand{\VariableTok}[1]{\textcolor[rgb]{0.00,0.34,0.68}{#1}}
\newcommand{\ControlFlowTok}[1]{\textcolor[rgb]{0.12,0.11,0.11}{\textbf{#1}}}
\newcommand{\OperatorTok}[1]{\textcolor[rgb]{0.12,0.11,0.11}{#1}}
\newcommand{\BuiltInTok}[1]{\textcolor[rgb]{0.39,0.29,0.61}{\textbf{#1}}}
\newcommand{\ExtensionTok}[1]{\textcolor[rgb]{0.00,0.58,1.00}{\textbf{#1}}}
\newcommand{\PreprocessorTok}[1]{\textcolor[rgb]{0.00,0.43,0.16}{#1}}
\newcommand{\AttributeTok}[1]{\textcolor[rgb]{0.00,0.34,0.68}{#1}}
\newcommand{\RegionMarkerTok}[1]{\textcolor[rgb]{0.00,0.34,0.68}{#1}}
\newcommand{\InformationTok}[1]{\textcolor[rgb]{0.69,0.50,0.00}{#1}}
\newcommand{\WarningTok}[1]{\textcolor[rgb]{0.75,0.01,0.01}{#1}}
\newcommand{\AlertTok}[1]{\textcolor[rgb]{0.75,0.01,0.01}{\textbf{#1}}}
\newcommand{\ErrorTok}[1]{\textcolor[rgb]{0.75,0.01,0.01}{\underline{#1}}}
\newcommand{\NormalTok}[1]{\textcolor[rgb]{0.12,0.11,0.11}{#1}}


\setlength{\emergencystretch}{3em}  % prevent overfull lines
\providecommand{\tightlist}{%
  \setlength{\itemsep}{0pt}\setlength{\parskip}{0pt}}



\institute{}
\titlegraphic{}

\usepackage{xcolor}
\newcommand\todonote[1]{\textcolor{red}{#1}}



% Subscript
\newcommand\sub[1]{_{#1}}
\newcommand\supsc[1]{^{#1}}

\title{Recruiting Profiles on HeroX}
\author{This version: \today}

\begin{document}
\maketitle


% Todo notes
%\usepackage[textsize=tiny]{todonotes}
%\newcommand\redmarginpar[1]{\marginpar{\footnotesize{\textcolor{red}{#1}}}}
%\listoftodos[Notes]
\clearpage

\begin{Shaded}
\begin{Highlighting}[]
\KeywordTok{rm}\NormalTok{(}\DataTypeTok{list=}\KeywordTok{ls}\NormalTok{())}
\KeywordTok{load}\NormalTok{(}\StringTok{"profiles.RData"}\NormalTok{)}
\end{Highlighting}
\end{Shaded}

\begin{Shaded}
\begin{Highlighting}[]
\KeywordTok{write.table}\NormalTok{(}\KeywordTok{iconv}\NormalTok{(profiles}\OperatorTok{$}\NormalTok{Q3[}\OperatorTok{-}\DecValTok{3}\NormalTok{], }\DataTypeTok{to=}\StringTok{"UTF-8"}\NormalTok{), }\DataTypeTok{quote=}\OtherTok{FALSE}\NormalTok{, }\DataTypeTok{col.names=}\OtherTok{FALSE}\NormalTok{, }\DataTypeTok{eol=}\StringTok{'}\CharTok{\textbackslash{}n\textbackslash{}n\textbackslash{}n\textbackslash{}n}\StringTok{'}\NormalTok{)}
\end{Highlighting}
\end{Shaded}

1 Innovation, Technology, Blockchain and Artificial Intelligence are
just some of the keywords that describe my passions.

Law student at the University of Milan, privacy officer and intellectual
property consultant. Winner of the Global Impact Competition Italy 2016,
studied exponential technologies for solving global grand challenges at
Singularity University. Creator of a project about Legal Smart
Contracts, took part in the Launchpad Program and won the HeroX Prize
Blockchain Social Impact Challenge .

Today I am working in UniCredit as part of Group Innovation, developing
and evaluating solution in the FinTech and RegTech space.

2 Currently I am student in Indian Institute of Technology, Hyderabad,
India doing my major in Mechanical and Aerospace Engineering. Also I am
a wantrepreneur as of now. I am passionate about designing new stuff
that could ease out human effort.

3 Since the age of 5, I always knew that I would work in a creative
capacity. I had visions of myself as an artist living in New York City,
a dream already fulfilled. I've created in the fashion realm, the
television and entertainment realm, the advertising and design realm,
the filmmaking realm, the architecture realm and the mom realm. All of
my creative experiences have lead me here, to figure out a way to
improve how those go through treatment for breast cancer. I believe in
creative minds and that a partnership in the medical realm will improve
that world for the better. This is what lead me to launch BC Pre-vis, a
pre-visualization program that helps patients understand what they can
expect for their own body prior to surgery.

I've worn many hats spanning from creative director, art director,
writer, designer, producer, product designer, architect, advisor and
entrepreneur but titles aside, I'm a creative problem solver who values
intelligence, talent and kindness above all else.

4 Short bio

I was raised wild and that turned me into a small entrepreneur as soon
as I graduated university. It was not a choice, but my only option at
the time. I have a Bachelor degree in Economics, specialized in
Management Accounting.

As a business owner in an emergent economy that was about to collide
with the world of corporations, I had to do a diversity of activities
and to reinvent my business continuously to keep my life on track. I
started with urban furniture production, continued with constructions,
travel and events production agency for business and executives groups,
and eventually with translations and interpreting services for
corporations.

Years after internet appeared in the developed world, I also discovered
it. It meant thousands of wires in the trees, Mirc and ads web pages.

In 2007 I discovered my talent for screenwriting and I entered Film
Directing university. I had to let it go after the first year as I
couldn't afford it anymore and had to choose between work and school.
That was the most painful decision I had to make. I felt I gave up on
myself.

I managed to keep and grow my business for 10 years and when I finally
got my first biggest contract, it all crushed when the client refused to
pay the rest of it. As it all happened right when the crisis hit, things
got very bad, from endless litigations to loosing everything, ending
somewhere on the margins of life.

The only industry that took me in and needed people like me was film
production. For years I did film locations, played the role of a fixer
for foreign TV and film productions and helped film studios and other
businesses to optimize their operations.

My Economics background lead me to the belief that out there in the
world some well informed people should already have build some bridges
between problems and solutions, so people with my kind of mindset, drive
and inner diversity may be in demand during these hard times. The
Americans was my first choice in the research. This is how I discovered
open innovation and started to solve problems and get awarded for my
solutions.

During all this time I was awarded for 5 solutions, most in the tech,
strategy and operational areas, developed a portfolio of over 40
solutions for both profit and non-profit organizations, and started We
Are Solvers project to raise awareness about the winning solvers
community.

On this road of redesigning myself and my life, I ended up as a
consultant for innovation management. Meeting HeroX uplifted the course
of things further. After my last win on their platform they offered me
the opportunity to work together on challenge design. This is how my
first transition from solution to challenge design in crowdsourcing
happened and it is a very interesting experience.

Today, I am back on my love for film, I write screenplays and started my
own web project in this area, and I continue to solve problems and
design challenges for worldwide organizations.

5 I have always had a passion for reducing waste and pollution in all
their forms, promoting reuse wherever possible. As a pre-teen I used to
collect discarded bicycle parts from the neighbourhood and build working
bikes out of them. As a young teen I found a discarded lawnmower,
brought it home and repaired it to serve as our family lawn mower for
many years. A little later I moved on to repairing cars and fixing up
old ones. I would restore old pieces of furniture and give them away.
I've been known to go on community trash cleanups, and served on the
sustainability council at a prior corporate employer. As a manager I
strived to best utilize each employee's human talents, providing them
with the best training and tools, and ample innovation opportunities.

Now, as an entrepreneur and freelance innovation strategist, I work on
solutions for global challenges. I started a company with trusted
colleagues to provide wearable environmental monitoring technology. More
recently I'm collaborating with an innovation community on ways to
tackle climate change by capturing CO2 for reuse and sequestration in a
widely distributed solution. In the past two years I have won awards on
12 crowdsourced innovation challenges helping many organizations with
finding new and better solutions to their problems. I like nothing more
than doing a deep dive into a new technical domain no matter what the
discipline.

6 I got my PhD in Entrepreneurial Studies at the school of hard knocks.
I was never good at school, and emotionally checked out somewhere around
grade 5. I kept going until Grade 12, but was more into the social
aspects. At age 25 I was quite messy, and some people challenged me with
a business opportunity. They convinced me to get into a
books/tapes/functions program. 11 years later, I had half a million cash
to launch my future. You can see 10 years of my life in my TEDx Talk,
``Poop Soup and the Inevitable Global Movement''. I am passionate about
taking hunger off the table as a barrier to advancement in life for
everyone, everywhere by decentralizing, demonetizing and democratizing
food, energy, water and education. We are well into our plan to create
an applied sciences centre where we bring all of agriculture and
agri-foods together with all of science and technology to develop
solutions towards absolute resource efficiency. (Zero Waste Agriculture)

7 I am professionally an Innopreneur, Idealist, teacher and Writer with
specific skills in Design and Realization, Engineering Drawing/AutoCAD
and Research/SPSS. I possess a Bachelors Degree in Technical Education
obtained from the University of Malawi and diverse trainings/courses in
innovation, project management, entrepreneurship and business management
obtained from University of Leeds, Malawi Center for research, 3Day
Start-up International and YALI Learns, respectively. My professional
achievements are as follows: I Innovated Engineering Drawing sets for
Malawi's Lilongwe Technical College (LTC) by utilizing the ideology of
Locally-Made-Instructional-Materials; I Proposed, designed and pioneered
Secondary-School Teacher Development Course for Maestros Leadership
Team; I Wrote a Student Mentorship book called Success and Creativity
currently being used by SOS children's village, Children of the Nation
(COTN), University of Malawi (Polytechnic) and LTC to train students in
innovation and creativity. I currently run an EduTech start-up Company
Called TECULES which is an acronym for Technologically Customized
Learning Systems. Our company focuses at building innovative learning
systems which are at par with latest technology to make learning
effective and simplify teaching methodologies (check www.tecules.com).
However, TECULES also secondarily engages in the development of Mobile
applications, Websites and soft-ware/database for companies,
organizations and related clients (www.solutions.tecules.com).

8 I have completed my MBA in Human Resources from Bangalore, India.

I have previously worked with HSBC and Amazon in Operations role where I
have had opportunities to work on Six sigma projects and facilitate
Process re-engineering.

Currently, I work as an HR Manager for a company based in Bangalore,
India. Through my job role, I connect employer branding with multiple
talent management and talent acquisition strategies. I am a geek at
heart and deeply enjoy combining my love for great workplace practices
with my penchant for business. These days, I mainly exercise my
nerdiness through online evaluation of the companies around the world to
uncover the best workplace practices.

\bibliography{library.bib}

\end{document}